\documentclass[a4paper]{article}
\usepackage{graphicx}
\usepackage[T1]{fontenc}
\setlength{\textwidth}{160mm}
\setlength{\oddsidemargin}{0mm}
\setlength{\evensidemargin}{0mm}
%\setlength{\textheight}{250mm}
%\setlength{\voffset}{-20mm}
%\usepackage{showframe}
\usepackage{xspace}
\usepackage{minted, xcolor}
\definecolor{bg}{rgb}{0.97,0.97,0.95}
\definecolor{bg_shell}{rgb}{0.95,0.95,1.00}

\newminted[shell]{bash}{bgcolor=bg_shell}


\title{\vspace{-20mm}Expertmaker Accelerator\\[1ex]\Large{Quick Install using the ``Project Skeleton'' Repository}}
\date{}
\begin{document}
\maketitle

\section*{Introduction and System Requirements}
The \texttt{accelerator\_project\_skeleton} project provides a simple
and convenient way to install the Accelerator.  This document lists
the necessary steps to set up the Accelerator using it.

The Accelerator will run on almost any hardware, from small laptops to
large multi-CPU rack servers.  It is assumed in this manual that the
computer is running Ubuntu~16.04~LTS or Debian 9.  The Accelerator
team is actively testing on Ubuntu, Debian, and FreeBSD, but the
Accelerator will most likely run on many other Linux distributions as well.


\section*{Installation}
There are three steps in the installation: resolve dependencies, clone
repository, and run the initiation script.  These steps will be
descibed next.
\subsection*{1. Dependencies}
The first step is to make sure that all software package dependencies
are met.  This command will install all required packages
\begin{shell}
sudo apt-get install build-essential python-dev python3-dev zlib1g-dev git virtualenv
\end{shell}
The installer requires nothing but \texttt{git}, \texttt{virtualenv},
and some \texttt{dev} packages in order to compile C-code.

\subsection*{2. Clone Repository}
Clone the \texttt{accelerator\_project\_skeleton} like this
\begin{shell}
git clone https://github.com/drougge/accelerator_project_skeleton.git
\end{shell}

\subsection*{3. Setup}
The Accelerator will now be installed \textsl{locally without any
administrator privileges}.  To continue, \texttt{cd} into the cloned
directory
\begin{shell}
cd accelerator_project_skeleton
\end{shell}
In this directory there is a file \texttt{init.py} that performs all
the installation steps.  It will work out-of-the-box, but for a
customised install it is recommended to read and modify this file
before continuing.  The next step is to run the script
\begin{shell}
./init.py
\end{shell}
This script will do a complete setup, and the next section provides
more information about the process.  After the script is finished, the
Acclerator installation is complete.  It could be run by issuing
\begin{shell}
cd accelerator
./daemon.py  
\end{shell}
The first time the Accelerator is run, it will compile some functions
written in the C programming language.  On some systems, this process
may generate a few warning messages, but that is okay.  Setup is now
complete.  \thispagestyle{empty}



\section*{Overview of the Installation}
The \texttt{accelerator\_project\_skeleton} script \texttt{init.py}
will setup virtual environments for Python2 and Python3.  In these
virtual environments, it will download and install some depending
packages, and \texttt{git clone} and install the
\texttt{accelerator-gzutil} library.  The Accelerator itself is
\texttt{git clone}d into a git submodule in the \texttt{accelerator}
directory.

The default configuration file is located in
\texttt{conf/framework.conf}.  This file is used to specify workdirs,
method directories, and more.  For more information, see the
Accelerator User's Reference Manual.



\section*{References}
\texttt{https://berkeman.github.io/pdf/acc\_manual.pdf}

\end{document}


