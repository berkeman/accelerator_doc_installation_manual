\documentclass[a4paper]{article}
\usepackage{graphicx}
\usepackage{color}
%% Replace serif font with (postscript) helvetica
%\usepackage[scaled]{helvet}
%\renewcommand*\familydefault{\sfdefault} %% Only if the base font of the document is to be sans serif
\usepackage[T1]{fontenc}
\setlength{\textwidth}{160mm}
\setlength{\oddsidemargin}{0mm}
\setlength{\evensidemargin}{0mm}
\setlength{\textheight}{250mm}
\setlength{\voffset}{-20mm}
%\usepackage{showframe}
\usepackage{xspace}
%\usepackage{pgffor}


\usepackage{xspace}
%%%  A pretty minted-environment for Python  %%%
\usepackage{minted, xcolor}
%\usemintedstyle{colorful}
\definecolor{bg}{rgb}{0.97,0.97,0.95}
\definecolor{bg_shell}{rgb}{0.95,0.95,1.00}
\newminted[python]{python}{bgcolor=bg, frame=lines}
\newminted[pythonBEG]{python}{bgcolor=bg, frame=topline}
\newminted[pythonMID]{python}{bgcolor=bg}
\newminted[pythonEND]{python}{bgcolor=bg, frame=bottomline}
\newminted[shellBEG]{text}{bgcolor=bg_shell, frame=topline}
\newminted[shellMID]{text}{bgcolor=bg_shell}
\newminted[shellEND]{text}{bgcolor=bg_shell, frame=bottomline}

\newminted[shell]{text}{bgcolor=bg_shell}

\newminted[text]{text}{bgcolor=bg, frame=lines}
\newminted[conf]{bash}{bgcolor=bg, frame=lines}


\title{\vspace{-0.5cm}Expertmaker Accelerator Installation Manual\vspace{-0.5cm}}
\date{}

\begin{document}
\maketitle

\section{Introduction}

This document covers installation of the Accelerator.  The Accelerator
will run on anything from a small laptop to a large multi-CPU rack
server.  In this manual, it is assumed that the computer is running
Ubuntu~16.04~LTS or Debian 9, but the Accelerator will run on many
other Linuxes as well as FreeBSD.  (Most of the development work has
been carried out on FreeBSD.)

The Accelerator is typically not installed ``system-wide''.  Instead,
it is instantiated as part of a project.  This manual describes how to
set up an empty project ``skeleton'', where the Accelerator is
included.

%For those who just need a quick reminder or like to figure things out
%by themselves, section~\ref{sec:quick} provides a minimalistic
%instruction.  A more comprehensive guide is found in
%section~\ref{sec:holdhand}.



\section{Installation}
\label{sec:quick}
This will install an Accelerator project skeleton on a machine running
Ubuntu~16.04 or Debian 9.
\begin{shell}
sudo apt-get install build-essential python-dev python3-dev zlib1g-dev git virtualenv

git clone https://github.com/drougge/accelerator_project_skeleton.git

cd accelerator_project_skeleton

./init.py
\end{shell}
The next step is to edit the configuration file
\texttt{conf/framework.conf} and change it to reflect the desired
setup.  The default configuration file assumes that some directories
exists, so if the file left unchanged, create these directories by
\begin{shell}
mkdir -p ~/accelerator/workdirs/TEST
\end{shell}
The Accelerator daemon is then started from inside the
\texttt{accelerator} directory like this
\begin{shell}
cd accelerator
./daemon.py  
\end{shell}
The first time the Accelerator is run, it will compile some functions
written in the C programming language.  On some systems, this process
may generate a few warnings, but that is okay.  Setup is now complete.
\thispagestyle{empty}



\section{Overview of the Installation}
The Accelerator project skeleton will setup virtual environments for
Python2 and Python3.  It will \texttt{git clone} the
\texttt{accelerator-gzutil} library and install in these virtual
environments.  The Accelerator itself is \texttt{git clone}d into a
git submodule in the \texttt{accelerator} directory.

The default configuration file is located in
\texttt{conf/framework.conf}.  This file is used to specify workdirs,
method directories, and more.  For more information, see the
Accelerator User's Reference Manual.


\section*{References}

\texttt{https://berkeman.github.io/accelerator\_doc\_users\_reference/accelerator\_manual.pdf}





\end{document}


