\documentclass[a4paper]{article}
\usepackage{graphicx}
\usepackage{color}
%% Replace serif font with (postscript) helvetica
%\usepackage[scaled]{helvet}
\renewcommand*\familydefault{\sfdefault} %% Only if the base font of the document is to be sans serif
\usepackage[T1]{fontenc}
\setlength{\textwidth}{160mm}
\setlength{\oddsidemargin}{0mm}
\setlength{\evensidemargin}{0mm}
\setlength{\textheight}{250mm}
\setlength{\voffset}{-20mm}
%\usepackage{showframe}
\usepackage{xspace}
%\usepackage{pgffor}


\usepackage{xspace}
%%%  A pretty minted-environment for Python  %%%
\usepackage{minted}
\usemintedstyle{colorful}
\definecolor{bg}{rgb}{0.97,0.97,0.95}
\definecolor{bg_shell}{rgb}{0.95,0.95,1.00}
\newminted[python]{python}{bgcolor=bg, frame=lines}
\newminted[pythonBEG]{python}{bgcolor=bg, frame=topline}
\newminted[pythonMID]{python}{bgcolor=bg}
\newminted[pythonEND]{python}{bgcolor=bg, frame=bottomline}
\newminted[shellBEG]{text}{bgcolor=bg_shell, frame=topline}
\newminted[shellMID]{text}{bgcolor=bg_shell}
\newminted[shellEND]{text}{bgcolor=bg_shell, frame=bottomline}
\newminted[shell]{text}{bgcolor=bg_shell, frame=lines}
\newminted[text]{text}{bgcolor=bg, frame=lines}
\newminted[conf]{bash}{bgcolor=bg, frame=lines}


\title{\vspace{-0.5cm}Expertmaker Accelerator Detailed Installation Manual\vspace{-0.5cm}}
\date{}

\begin{document}
\maketitle

\section{Introduction}

This document covers installation of the Accelerator on a single
computer instance.  Due to its small footprint and efficient design,
the Accelerator will run on anything from a small laptop to a large
multi-CPU rack server.  It is assumed that the computer is running
Ubuntu~16.04~LTS, but the Accelerator will run without difficulties on
many other Linuxes as well as FreeBSD.  (In fact, most of the
development work is carried out on FreeBSD boxes.)

The Accelerator is typically not installed ``system-wide''.  Instead,
it is instantiated as part of a project.  This manual describes how to
set up an empty project ``skeleton'', where the Accelerator is
included.  The last section is devoted to a simple run-test of the
installation.

For those who just need a quick reminder or like to figure things out
by themselves, section~\ref{sec:quick} provides a minimalistic
instruction.  A more comprehensive guide is found in
section~\ref{sec:holdhand}.



\section{Quick Setup}
\label{sec:quick}
This will install an Accelerator project skeleton on a machine running
Ubuntu~16.04.
\\
\begin{shell}
  sudo apt-get install python-ujson python3-ujson python-cffi python-numpy
  git clone git@github.corp.ebay.com:cdrougge/accelerator_project_skeleton.git
  cd accelerator_project_skeleton
  git submodule init
  git submodule update
  mkdir -p ~/workdirs/test
\end{shell}
\\ The next step is then to edit the configuration file
\texttt{conf/framework.conf} and change it to reflect the desired
setup.  Done.



\section{Detailed Setup}
\label{sec:holdhand}
This section provides setup instructions with more detailed comments.

\subsection{Dependencies}
The Accelerator relies on a few packages not installed by default in
Ubuntu~16.04.  These packages may be installed by issuing
\\
\begin{shell}
  sudo apt-get install python-ujson python3-ujson python-cffi python-numpy
\end{shell}
\\
This should be sufficient for Ubuntu~16.04.  Other distributions
may require other packages, such as \texttt{python-dev} and
\texttt{zlib1g-dev}.


\subsection{A New Accelerator Project}
First, check out an empty project skeleton from the repository
\\
\begin{shellBEG}
  git clone <location>/accelerator_project_skeleton.git
\end{shellBEG}
\\
and \texttt{cd} down into the directory
\\
\begin{shellMID}
  cd accelerator_project_skeleton
\end{shellMID}
\\
The Accelerator itself is included as a git submodule.  Git submodules
are initiated by issuing
\\
\begin{shellMID}
  git submodule init
\end{shellMID}
\\
This will print which repository the Accelerator code will be cloned
from, and
\\
\begin{shellEND}
  git submodule update
\end{shellEND}
\\
will actually clone the Accelerator repository into the directory
\texttt{accelerator}.

\noindent Make sure this is the case.
\\
\begin{shell}
  ls -l
\end{shell}
\\
will show three directories like this
\\
\begin{text}
drwxrwxr-x 8 ab ab 4096 dec  8 15:35 accelerator
drwxrwxr-x 3 ab ab 4096 dec  9 15:44 analysis
drwxrwxr-x 2 ab ab 4096 dec  9 15:03 conf
\end{text}
\\
Here, \texttt{accelerator},
holds the Accelerator code, \texttt{conf} holds the
configuration files, and \texttt{analysis} is the default location for
user designed methods in the project.

The setup provides independent source control of the Accelerator code
base and the project using it.  Since submodule commit states may be
committed into the base repository, it is possible to keep track of
which version of the Accelerator that was used at any given
(committed) state of a project.


\subsection{A Workdir}

The next step is to create a workdir.  This is the directory where all
job data and results are stored.
\\
\begin{shell}
  mkdir -p ~/workdirs/test
\end{shell}
\\
This location is fine for simple testing and is consistent with the
supplied configuration file.  In general, workdirs may be located
anywhere in the file system.

\subsection{The Configuration File}

A configuration file is supplied in the skeleton repository, and it
may be used without changes.  The file specifies, among other things,
where in the file system the Accelerator is allowed to write, so it is
a good idea to have a look at it.
\\
\begin{shell}
  cat conf/framework.conf
\end{shell}



\begin{conf}
# Define a workdir named "test" with 17 slices.  Each slice will use
# one CPU core, so adjust according to target platform capability.
workspace=test:${HOME}/workdirs/test:17

# You can only have one main workspace.
# All built jobs end up there.
main_workspace=test

# Methods are imported from these directories (comma separated)
method_directories=analysis,default_analysis

# daemon log
logfilename=${HOME}/daemon.log

# python versions to use
# (the left side here is what you put on the right side in methods.conf)
py2=/usr/bin/python2.7
py3=/usr/bin/python3.5
\end{conf}
\\
The \texttt{method\_directories}
specify the two directories \texttt{default\_analysis}, which is where
the Accelerator's ``built-in'' methods reside; and \texttt{analysis},
where user created methods are located by default.
Note that the configuration file is part of the project repository,
and thus independent of the accelerator.

\subsection{Trying it Out}
The Accelerator daemon is always executed from the
\texttt{accelerator} directory
\\
\begin{shell}
  cd accelerator
  ./daemon.py
\end{shell}
\\
If everything went well, the daemon should now start.  Setup is now
complete.







\end{document}


